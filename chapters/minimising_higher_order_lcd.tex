\chapter{Counterdiabatic driving as a cost function}\label{chap:5_cd_as_costfunc}

In the previous chapter I discussed the idea of combining approximate (local) counterdiabatic driving (\acrref{LCD}) and quantum optimal control in order to improve upon both approaches in maintaining a system in its instantaneous eigenstate as it changes  

\section{The idea}

\begin{equation}\label{eq:AGP_integral_1}
\mathcal{I}_1 = \int_0^\tau dt^\prime \Big[\bra{\psi_g(t^\prime)} \Gamma^2(t^\prime) \ket{\psi_g(t^\prime)}  - (\bra{\psi_g(t^\prime)} \Gamma(t^\prime) \ket{\psi_g(t^\prime)})^2\Big]^{1/2},
\end{equation}

\begin{equation}\label{eq:Gamma_def}
\Gamma(t) = \gamma(t) \left( \sy_1 \sx_2 + \sx_1 \sy_2 \right),
\end{equation}

\begin{equation}\label{eq:AGP_integral_2}
\mathcal{I}_2 = \int_0^\tau dt^\prime |\gamma(t^\prime)|,
\end{equation}

There are several metrics that can be used to quantify the impact of a particular approximation of the \acrref{CD} on the dynamics of a system and its chance to be excited out of the instantaneous eigenstate. Here we explore two options: the peak amplitude of the counterdiabatic drive and its integral across the entire evolution of th

\subsection{Minimising peak amplitudes}

\subsection{Minimising integrals of the drive}

\reminder{mention Ewen's paper!}