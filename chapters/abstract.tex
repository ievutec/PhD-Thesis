\chapter{Abstract}\label{chap:abstract}

Adiabatic protocols are employed across a variety of quantum technologies, from implementing state preparation and individual operations that are building blocks of larger devices, to higher-level protocols in quantum annealing and adiabatic quantum computation. The problem of speeding up these processes has garnered a large amount of interest, resulting in a menagerie of approaches, most notably quantum optimal control and shortcuts to adiabaticity. The two approaches are complementary: optimal control manipulates control fields to steer the dynamics in the minimum allowed time, while shortcuts to adiabaticity aims to retain the adiabatic condition upon speed-up. We outline a new method that combines the two methodologies and takes advantage of the strengths of each. The new technique improves upon approximate local counterdiabatic driving with the addition of time-dependent control fields. We refer to this new method as counterdiabatic optimized local driving (COLD) and we show that it can result in a substantial improvement when applied to annealing protocols, state preparation schemes, entanglement generation, and population transfer on a lattice. We also demonstrate a new approach to the optimization of control fields that does not require access to the wave function or the computation of system dynamics. COLD can be enhanced with existing advanced optimal control methods and we explore this using the chopped randomized basis method and gradient ascent pulse engineering.