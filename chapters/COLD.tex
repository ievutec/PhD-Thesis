\part{Counterdiabatic optimised local driving}

\chapter{The COLD method}\label{chap:4_COLD}

\epigraph{A good idea has a way of becoming simpler and solving problems other than that for which it was intended.}{Robert Tarjan}

In my experience, one of the best ways to find a solution is to first focus on defining the problem you're trying to solve. In fact, the quote at the start of this chapter refers to the concept of the `rising sea' in mathematics, first coined in a longer and more poetic quote by Alexander Grothendieck \cite{mclarty_grothendieck_nodate}. The idea, roughly, is that a series of trivial steps and insights, each of which doesn't seem to be all that interesting or important, combine over time to reveal an emerging non-trivial result. To me, this is exemplified quite nicely by the \acrref{COLD} method, the conception of which follows from rather trivial observations about the \acrref{LCD} approach (discussed in detail in Sec.~\ref{sec:2.4.1_LCD}), its shortcomings, and the practical requirements surrounding the controllability of realistic quantum systems. 

In short: the problem that \acrref{LCD} sets out to solve is the suppression of non-adiabatic losses and deformations in quantum systems that arise from fast changes in a time-dependent driving Hamiltonian. Slow (\@i.e.~ adiabatic) driving is a solution in theory, but in practice slow transformations lead to 

\section{The method}

The \acrref{COLD} approach 

We begin, as does \acrref{LCD}, with the 

\begin{equation}\label{eq:expandedH}
H_{\rm COLD}(\lambda,\betabb) = H_0(\lambda) + {\boldsymbol \alpha}(\lambda,\betabb) \mathcal{O}_{\rm LCD} + \betabb(\lambda) \mathcal{O}_{\rm opt}.
\end{equation}

\section{The constraints}

\section{Optimal control toolbox}

\subsection{COLD-CRAB}

Mention differentiability as an advantage! It's a thing.

Also mention how I accidentally re-derived dCRAB.

\subsection{COLD-GRAPE}