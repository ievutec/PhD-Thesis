\part{Counterdiabatic optimised local driving}

\chapter{COLD}\label{chap:4_COLD}

\epigraph{A good idea has a way of becoming simpler and solving problems other than that for which it was intended.}{Robert Tarjan}

In Ch.~\ref{chap:2_adiabaticity} we established that adiabatic evolution of a quantum system requires very long timescales without which it experiences non-adiabatic excitations out of its instantaneous eigenstate(s). With the dual motivation of preserving adiabaticity, \@i.e.~ enforcing that the system stay in its instantaneous eigenstates, and speeding up its evolution

\iffalse
In Ch.~\ref{chap:2_adiabaticity} I discussed the concept of quantum adiabaticity: roughly, the idea that a system evolving under a slowly changing time-dependent Hamiltonian will remain in its instantaneous eigenstate(s) throughout the evolution, given no degeneracies are present throughout the process. Such adiabatic protocols are incredibly useful for developing quantum technologies, finding applications in everything from state preparation \cite{dimitrova_many-body_2023} and quantum gate synthesis \cite{pelegri_high-fidelity_2022} to quantum thermodynamics\cite{campo_more_2014} and solving combinatorics problems via quantum annealing \cite{ebadi_quantum_2022}. However, the condition of `slow' evolution quantified by the bound given in Eq.~\eqref{eq:adiabatic_criterion}, is often difficult to satisfy as quantum systems are volatile and decohere quickly due to interactions with the environment and in settings like quantum computing, time-efficiency tends to be quite important. Unfortunately, as the driving speed of the Hamiltonian is increased, the probability of a system getting excited out of its instantaneous eigenstate increases along with it, leading to unwanted non-adiabatic losses which render the protocol ineffective. 

In order to solve the issue of long timescales, many methods have been proposed, grouped under the umbrella of Shortcuts to Adiabaticity (\acrref{STA}) \cite{guery-odelin_shortcuts_2019}. A universal approach is provided by Counterdiabatic Driving (\acrref{CD}), which aims to completely suppress the non-adiabatic effects generated by finite-time change in the driving Hamiltonian by applying an external field. However, as discussed in Sec.~\ref{sec:2.3_CD}, it is often both difficult to derive the exact form of the counterdiabatic drive and to engineer it for a particular instance of the problem as it requires knowledge of the full eigenspectrum of the system at each instant of time. These shortcomings can be addressed by instead working with approximations of the exact \acrref{CD}, like Local Counterdiabatic Driving (\acrref{LCD}), presented in Sec.~\ref{sec:2.4.1_LCD} or the nested commutator approach from Sec.~\ref{sec:2.4.2_nested_commutators}. \reminder{finish this}
\fi
This chapter will present a new method, known as Counterdiabatic Optimised Local Driving (\acrref{COLD}), which aims to circumvent the shortcomings of \acrref{LCD} while retaining its advantages of locality and simplicity. It does this by combining ideas from Quantum Optimal Control Theory (\acrref{QOCT}), which were covered extensively in Ch.~\ref{chap:3_Quantum_Optimal_control}

\section{Combining counterdiabatic driving and optimal control}

The \acrref{COLD} approach begins with the observation that the counterdiabatic schedule will depend on the driving path of the original Hamiltonian $H(\lambda)$ for which it is constructed, where $\lambda$ is a parameter which captures the time-dependence. This can be seen, for example, in the fact that an exact \acrref{CD} protocol (Eq.~\eqref{eq:CD_Hamiltonian}) implements a drive proportional to the adiabatic gauge potential (\acrref{AGP}) operator $\AGP{\lambda}$ scaled by the speed of the change in the driving parameter $\dotlambda$. The off-diagonal matrix elements of the \acrref{AGP} operator, responsible for the non-adiabatic effects and  given by Eq.~\eqref{eq:AGP_adiabatic_basis}, depend both on the matrix elements of $\dlambda H(\lambda)$ and the instantaneous eigenenergies and eigenstates of $H(\lambda)$. Thus, for a Hamiltonian made up of a set of operators $\mathcal{O}_{H}$ parameterised by a set of coefficients $\hbb = \{h_i\}_{i = 1, ..., N_h}$  and the parameter $\lambda$, which captures its time-dependence, 

The full form of the \acrref{AGP}, given by Eq.~\eqref{eq:AGP_adiabatic_basis}, depends on both the instantaneous eigenstates of $H(\lambda)$ and the matrix 


\begin{equation}\label{eq:expandedH}
H_{\rm COLD}(\lambda,\betabb) = H_0(\lambda) + {\boldsymbol \alpha}(\lambda,\betabb) \mathcal{O}_{\rm LCD} + \betabb(\lambda) \mathcal{O}_{\rm opt}.
\end{equation}

\section{The constraints}

\section{Optimal control toolbox}

When it comes to the optimal control component of \acrref{COLD}, it i 

\subsection{COLD-CRAB}

Mention differentiability as an advantage! It's a thing.

Also mention how accidentally re-deriving dCRAB.

\subsection{COLD-GRAPE}

Modified version of \acrref{GRAPE}: not the gradient-based approach, but rather one which 

spline, interpolation